\begin{subsection} {Ion Snapshot}
  \begin{subsubsection} { Terminology: Snapshot and Frame }
    A user might want to record each ion's position and velocity at various times in the simulation. Let's call these information \emph{snapshots}. Unlike what it litarally sounds, a snapshot couldeither include ion's instaneous positions and velocities information at a single time step, called \emph{frame}, or a consecutive series of frames.
    For my implementation of ion snapshots, every snapshot contains same number of frames. So to specify how ion snapshots should be stored, a user need to provide the start time $t_i$ for $i^{th}$ snapshot with $i=0, 1, ... N_s$ and number of frames $N_{f}$ in each snapshot.
  \end{subsubsection}


  \begin{subsubsection} {Syntax}
    The statement to instruct \textsc{ProtoMol} to write snapshots to files on disk is
    \begin{lstlisting}[emph=IonSnapshot]
      IonSnapshot filename\end{lstlisting}
    where \textbf{IonSnapshot} points to a lua file that defines these snapshots. An example snapshot definition file is presented below,
    \begin{lstlisting}[language=LUA]{Example file of snapshot definition}
      start_time = {}

      for i=1, 50 do
        start_time[i] = (i-1) * 2e-2
      end

      ss = {
        num_frame = 250, 
        start = start_time 
      }

      dir = ./\end{lstlisting}
    which should always declare a table variable \emph{ss}, with the fields explained in Tab. 
    \begin{table}[h!]
      \centering
      \begin{tabular}{l  l  l  p{7cm} } \toprule
        Field & Type &  Unit & Meaning\\ \midrule
        num\_frame & integer & 1 & $N_f$, if set to -1, then $N_f$ set to number of steps  in the longest secular periods in $x,y,z$ three directions \\ 
        start & table & sec & $t_i$, i.e. the time of the first frame \\
        dir & string & N/A & The name of directory where snapshot files are stored\\
        \bottomrule
      \end{tabular}
      \caption{Snapshot Definition}
      \label{tab:ion_snapshot_def}
    \end{table}
  \end{subsubsection}

\begin{subsubsection} {Output Directory Structure}
All the snapshot files are stored under the same directory, specified by the \textbf{dir} keywords in the definition file. They are named by the pattern snapshot\_$i$.hd5, where $i$ ranges from $0$ to $N_s-1$.
\tikzstyle{every node}=[draw=black,thick,anchor=west]
\tikzstyle{selected}=[draw=red,fill=red!30]
\tikzstyle{optional}=[dashed,fill=gray!50]
\begin{tikzpicture}[%
  grow via three points={one child at (0.5,-0.7) and
  two children at (0.5,-0.7) and (0.5,-1.4)},
  edge from parent path={(\tikzparentnode.south) |- (\tikzchildnode.west)}]
  \node {dir}
    child { node {snapshot\_$0$.hd5} }
    child { node {snapshot\_$1$.hd5} }
    child { node {...} }
    child { node {snapshot\_$(N_s-1)$.hd5} };
\end{tikzpicture}
\end{subsubsection}


\begin{subsubsection} {Snapshot File Structure}
The snapshot files are stored in HDF5 format, for its compact size and support by major programming languages, including C/C++/Python/Matlab. For an overview of HDF5 file, please visit its official website at \url{http://www.hdfgroup.org/HDF5/}.

Every HDF5 file has a hierachical data structure, very similar to the file system structures in an operating system. In addition to $N_f$ consecutive frames, a \textit{config} header is also included which consists of auxillary information intended to make the snapshot file self-inclusive.
\tikzstyle{every node}=[draw=black,thick,anchor=west]
\tikzstyle{selected}=[draw=red,fill=red!30]
\tikzstyle{optional}=[dashed,fill=gray!50]
\begin{tikzpicture}[%
  grow via three points={one child at (0.5,-0.7) and
  two children at (0.5,-0.7) and (0.5,-1.4)},
  edge from parent path={(\tikzparentnode.south) |- (\tikzchildnode.west)}]
  \node {/}
    child { node {config} 
      child { node {start} }
      child { node {fps} }
      child { node {numFrame} }
      child { node {numAtom} }
      child { node {numFrameperMM} }
    }
    child [missing] {}				
    child [missing] {}				
    child [missing] {}				
    child [missing] {}				
    child [missing] {}	
    child { node {frame0} 
      child {node {$x_1, y_1, z_1, v_{x,1}, v_{y,1}, v_{z,1}$} } 
      child {node {$x_2, y_2, z_2, v_{x,2}, v_{y,2}, v_{z,2}$} } 
      child {node {...} } 
      child {node {$x_n, y_n, z_n, v_{x,n}, v_{y,n}, v_{z,n}$} }
    }
    child [missing] {}				
    child [missing] {}				
    child [missing] {}				
    child [missing] {}			
    child { node {frame1} 
      child {node {$x_1, y_1, z_1, v_{x,1}, v_{y,1}, v_{z,1}$} } 
      child {node {$x_2, y_2, z_2, v_{x,2}, v_{y,2}, v_{z,2}$} } 
      child {node {...} } 
      child {node {$x_n, y_n, z_n, v_{x,n}, v_{y,n}, v_{z,n}$} }
    }
    child [missing] {}				
    child [missing] {}				
    child [missing] {}				
    child [missing] {}			
    child { node {frame2} }
    child { node {...} }
    child { node {frameN} };
\end{tikzpicture}
\end{subsubsection}
\end{subsection}
