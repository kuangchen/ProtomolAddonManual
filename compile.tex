\begin{section} {Compiling \textsc{ProtoMol}}
The compilation of original \textsc{ProtoMol} source code do not require any additional external programming libraries. The modified version however, requires the following libraries, to be installed first. 
\begin{itemize}{}{}
\item Lua5.2
\item HDF5
\end{itemize}
On Ubuntu/Linux systems, these packages can be installed by \textsc{apt-tools}. On Windows, these packages can be found at \url{http://code.google.com/p/luaforwindows/}, and \url{http://www.hdfgroup.org/HDF5/release/obtain5.html}. As for the compiler, because some new features introduced in C++11 are used, a compatible C++ compiler should be used. The compilation process have been simplified by \textsc{CMake} tool. On Linux machine, installation is accomplished by the following commands, 
\lstset{language=bash} 
\begin{lstlisting}
# Remove previous CMakeCache
rm -rf CMakeCache.txt CMakeFiles

# Instructs ProtoMol to have lapack built
cmake -DBUILD_LAPACK=ON -DBUILD_LAPACK_TYPE=lapack 

# Tell cmake to generate Makefiles
cmake .

# Install ProtoMol into system folder, as super user
sudo make install
\end{lstlisting}
\end{section}
